\documentclass[12pt, notitlepage]{article}   	% use "amsart" instead of "article" for AMSLaTeX format
\usepackage{geometry}                		% See geometry.pdf to learn the layout options. There are lots.
\geometry{a4paper}                   		% ... or a4paper or a5paper or ... 
%\geometry{landscape}                		% Activate for rotated page geometry
\usepackage[parfill]{parskip}    		% Activate to begin paragraphs with an empty line rather than an indent
\usepackage{graphicx}				% Use pdf, png, jpg, or eps§ with pdflatex; use eps in DVI mode
								% TeX will automatically convert eps --> pdf in pdflatex

\usepackage{hyperref}


%SetFonts

\usepackage[T1]{fontenc}
\usepackage[utf8]{inputenc}

\usepackage{tgbonum}

%SetFonts

\title{
	\textbf{
		Readings
	} \\
	\large Plant Physiological Ecology \\
	\large Spring 2021
}

\date{\vspace{-5ex}}

\begin{document}

{\fontfamily{phv}\selectfont %select helvetica (code = phv)

\maketitle

**Please contact Dr. Smith if you have trouble accessing the articles**

**Note: this file will be updated to account for changes to the schedule**

\section*{Week of January 25}
\textit{Classical Literature Tuesday - Jan 26} \par
Chapin FS. 2003. Effects of Plant Traits on Ecosystem and Regional Processes: 
a Conceptual Framework for Predicting the Consequences of Global Change. 
Annals of Botany 91: 455–463. \par
\url{https://academic.oup.com/aob/article/91/4/455/213070}

\textit{Recent Literature Thursday - Jan 28} \par
Reich PB. 2014. The world-wide ‘fast–slow’ plant economics spectrum: a traits manifesto. 
Journal of Ecology 102: 275–301. \par
\url{https://besjournals.onlinelibrary.wiley.com/doi/10.1111/1365-2745.12211}
\par

\section*{Week of February 1}
\textit{Classical Literature Tuesday - Feb 2} \par
Von Caemmerer S, Farquhar GD. 1981. Some relationships between the biochemistry of 
photosynthesis and the gas exchange of leaves. Planta 153: 376–387. \par
\url{https://link.springer.com/article/10.1007/bf00384257}

\textit{Recent Literature Thursday - Feb 4} \par
Smith NG, Dukes JS. 2018. Drivers of leaf carbon exchange capacity across biomes at 
the continental scale. Ecology 99: 1610–1620. \par
\url{https://esajournals.onlinelibrary.wiley.com/doi/full/10.1002/ecy.2370}

\section*{Week of February 8 and 15}
\textit{Classical Literature Tuesday - Feb 9} \par
Boardman NK. 1977. Comparative photosynthesis of sun and shade plants. 
Annual review of plant physiology 28: 355–377. \par
\url{https://www.annualreviews.org/doi/10.1146/annurev.pp.28.060177.002035}

\textit{Recent Literature Thursday - Feb 18} \par
Bennie J., Davies T.W., Cruse D. & Gaston K.J. 2016. 
Ecological effects of artificial light at night on wild plants. 
Journal of Ecology 104, 611–620. \par
\url{https://besjournals.onlinelibrary.wiley.com/doi/10.1111/1365-2745.12551}

\section*{Week of February 22}
\textit{Classical Literature Tuesday - Feb 23} \par
Atkin OK and Tjoelker M. 2003. Thermal acclimation and the dynamic response of plant 
respiration to temperature. Trends in Plant Science 8: 343–351. \par
\url{https://www.sciencedirect.com/science/article/pii/S1360138503001365}

\textit{Recent Literature Thursday - Feb 25} \par
Slot, M. and Winter, K. (2017), In situ temperature response of photosynthesis of 42 tree 
and liana species in the canopy of two Panamanian lowland tropical forests with 
contrasting rainfall regimes. New Phytologist 214: 1103-1117. \par
\url{https://nph.onlinelibrary.wiley.com/doi/full/10.1111/nph.14469}

\section*{Week of March 1}
\textit{Classical Literature Tuesday - Mar 2} \par
Chaves MM, Pereira JS, Maroco J, et al. 2002. How Plants Cope with Water Stress 
in the Field? Photosynthesis and Growth. Annals of Botany 89: 907–916. \par
\url{https://academic.oup.com/aob/article/89/7/907/151103}

\textit{Recent Literature Thursday - Mar 4} \par
Li X., Blackman C.J., Choat B., Duursma R.A., Rymer P.D., Medlyn B.E. & Tissue D.T. 2018.
Tree hydraulic traits are coordinated and strongly linked to climate‐of‐origin across a rainfall gradient.
Plant, Cell & Environment 41: 646–660. \par
\url{https://onlinelibrary.wiley.com/doi/full/10.1111/pce.13129}

\section*{Week of March 8}
\textit{Classical Literature Tuesday - Mar 9} \par
Bazzaz FA. 1990. The response of natural ecosystems to the rising global CO2 levels. 
Annual review of ecology and systematics 21: 167–196. \par
\url{https://www.annualreviews.org/doi/10.1146/annurev.es.21.110190.001123}

\textit{Recent Literature Thursday - Mar 11} \par
Swann A.L.S., Hoffman F.M., Koven C.D. & Randerson J.T. (2016) Plant responses to 
increasing CO2 reduce estimates of climate impacts on drought severity. 
Proceedings of the National Academy of Sciences 113: 10019–10024. \par

\section*{Week of March 15}
\textit{Classical Literature Tuesday - Mar 16} \par
LeBauer, D. S. and Treseder, K. K. (2008), Nitrogen limitation of net primary productivity
in terrestrial ecosystems is globally distributed. Ecology, 89: 371-379. \par
\url{https://esajournals.onlinelibrary.wiley.com/doi/full/10.1890/06-2057.1}

\textit{Recent Literature Thursday - Mar 18} \par
Delpiano C.A., Prieto I., Loayza A.P., Carvajal D.E. & Squeo F.A. (2020) 
Different responses of leaf and root traits to changes in soil nutrient availability do 
not converge into a community-level plant economics spectrum. Plant and Soil 450: 463–478. \par
\url{https://link.springer.com/article/10.1007/s11104-020-04515-2}

\section*{Week of March 22}
\textit{Classical Literature Tuesday - Mar 23} \par
Mooney HA. 1972. The carbon balance of plants. 
Annual review of ecology and systematics 3: 315–346. \par
\url{https://www.annualreviews.org/doi/10.1146/annurev.es.03.110172.001531}

\textit{Recent Literature Thursday - Mar 25} \par
TBD \par

\section*{Week of March 29}
\textit{Classical Literature Tuesday - Mar 30} \par
Givnish TJ. 2002. Adaptive significance of evergreen vs. deciduous leaves: 
solving the triple paradox. Silva fennica 36: 703–743. \par
\url{https://silvafennica.fi/article/535}

\textit{Recent Literature Thursday - Apr 1} \par
Santini, B.A., Hodgson, J.G., Thompson, K., Wilson, P.J., Band, S.R., Jones, G., 
Charles, M., Bogaard, A., Palmer, C. and Rees, M., 2017. 
The triangular seed mass–leaf area relationship holds for annual plants and is determined by habitat productivity. 
Functional Ecology, 31: 1770-1779.
\url{https://besjournals.onlinelibrary.wiley.com/doi/full/10.1111/1365-2435.12870}

\section*{Week of April 5}
\textit{Classical Literature Tuesday - Apr 6} \par
Grime JP. 1977. Evidence for the Existence of Three Primary Strategies in Plants and Its 
Relevance to Ecological and Evolutionary Theory. 
The American Naturalist 111: 1169–1194. \par
\url{https://www.jstor.org/stable/2460262}

\textit{Recent Literature Thursday - Apr 8} \par
TBD

\section*{Week of April 12}
\textit{Classical Literature Tuesday - Apr 13} \par
Wright DP, Scholes JD, Read DJ. 1998. Effects of VA mycorrhizal colonization on 
photosynthesis and biomass production of Trifolium repens L. 
Plant, Cell and Environment 21: 209–216. \par
\url{https://onlinelibrary.wiley.com/doi/10.1046/j.1365-3040.1998.00280.x}

\textit{Recent Literature Thursday - Apr 15} \par
TBD

\section*{Week of April 19}
\textit{Classical Literature Tuesday - Apr 20} \par
Field CB, Lobell DB, Peters HA, Chiariello NR. 2007. Feedbacks of Terrestrial Ecosystems 
to Climate Change. Annual Review of Environment and Resources 32: 1–29. \par
\url{https://www.annualreviews.org/doi/10.1146/annurev.energy.32.053006.141119}

\textit{Recent Literature Thursday - Apr 22} \par
No reading \par
%\url{https://onlinelibrary.wiley.com/doi/full/10.1111/j.1365-2486.2012.02797.x}

\section*{Week of April 26}
\textit{Classical Literature Tuesday - Apr 27} \par
No reading \par
%\url{https://www.annualreviews.org/doi/10.1146/annurev.energy.32.053006.141119}

\textit{Recent Literature Thursday - Apr 29} \par
No reading \par
%\url{https://onlinelibrary.wiley.com/doi/full/10.1111/j.1365-2486.2012.02797.x}

} %end font selection

\end{document}