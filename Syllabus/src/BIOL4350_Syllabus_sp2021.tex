\documentclass[12pt, notitlepage]{article}   	% use "amsart" instead of "article" for AMSLaTeX format
\usepackage{geometry}                		% See geometry.pdf to learn the layout options. There are lots.
\geometry{a4paper}                   		% ... or a4paper or a5paper or ... 
%\geometry{landscape}                		% Activate for rotated page geometry
\usepackage[parfill]{parskip}    		% Activate to begin paragraphs with an empty line rather than an indent
\usepackage{graphicx}				% Use pdf, png, jpg, or eps§ with pdflatex; use eps in DVI mode
								% TeX will automatically convert eps --> pdf in pdflatex

\usepackage{hyperref}
		
%SetFonts

\usepackage[T1]{fontenc}
\usepackage[utf8]{inputenc}

\usepackage{tgbonum}

%SetFonts

\title{
	\textbf{
		BIOL 4350
	} \\
	\large Plant Physiological Ecology \\
	\large Spring 2021
}

\date{\vspace{-5ex}}

\begin{document}

{\fontfamily{phv}\selectfont %select helvetica (code = phv)

\maketitle

\section{Course Description}
Students in this course will learn the fundamentals of plant physiology through 
an ecological lens. The course will focus on plant responses to environmental conditions 
across multiple spatial and temporal scales. The course will cover plant, water, carbon, 
and nutrient relations in natural and managed systems across multiple ecological scales. 
Students will be evaluated on their 
ability to discuss and disseminate ecophysiological topics.

\subsection{Class Time and Location}
Tuesdays and Thursdays 9:30-10:50

Biology Building (BIOL) Room 101 or 
synchronous online via Zoom (see Mode of Instruction section below).

\subsection{Instructor}
Dr. Nick Smith \par
Biology Building (BIOL) Room 215 \par
806-834-7363 \par
nick.smith@ttu.edu \par
\textit{Meetings by appointment}

\subsection{Teaching Assistant}
Mr. Evan Perkowski \par
evan.a.perkowski@ttu.edu \par
\textit{Meetings by appointment}

\subsection{Recommended Texts}
Plant Physiological Ecology (2nd Edition; 2008) by Lambers, Chapin, and Pons \par
The book can be accessed from Springer here: 
\url{https://www.springer.com/us/book/9780387783406}. Click on "Access this title on 
SpringerLink." It can also be accessed through the TTU library. \par
Plant Physiology and Development (6th Edition) by Taiz, Ziegler, Moller, and Murphy

\section{Mode of Instruction}
All instruction will be given in a synchronously.
The course will be taught in a hybrid mode. Students will have the option to either
attend class in person (BIOL 101) or online via Zoom. In-person and online attendees
will be expected to partipate fully in the class (see expectations and grading below).
All students, regardless of how they attend, will be evaluated similarly.
Students wanting to attend online will need to download Zoom to their personal computer
(\url{https://zoom.us/}).
Students will need to have access to a webcam and microphone for Zoom delivery.
The Zoom meeting ID and password will be emailed to the class by Dr. Smith.

\section{Contingency Statement}
If Texas Tech University campus operations are required to change because of health 
concerns related to the COVID-19 pandemic, it is possible that this course will move to 
a fully online delivery format. Should that be necessary, students will need to have 
access to a webcam and microphone for remote delivery of the class. Additionally, students 
will need to have access to Zoom software.

\section{Course Materials}
All course materials, including lecture slides, readings, activities, and code 
will be posted to a GitHub repository for the course.
The primary repository address is
\url{https://github.com/SmithEcophysLab/ecophys_sp2021}.

\section{Learning Objective}
This course will broadly focus on understanding the role that plant physiological 
processes play in driving ecological responses across multiple scales from the individual 
to the globe. Class activities will be based on discussion and dissemination of ideas, 
including classic and recent scientific literature. 
Topics will be flexible and modified to match student interests where possible.

\section{Attendance Policy}
Attendance is strongly recommended. 
Much of the graded content will be completed in class and students will not be permitted
to complete this material outside of class. See course assessments below for details.
Makeups will not be granted.

\section{Course Assessment}
\subsection{\textit{Participation and Engagement}}
Being an active and engaged participant in the class will benefit your understanding
of material as well as your peers'. Examples include asking questions, providing feedback,
and facilitating discussion. Participation and engagement of each student will be monitored
during each class period.

\subsection{\textit{Mini-quizzes}}
Short “quizzes” will be given in class each week (typically on Thursdays). 
These quizzes will be used to stimulate discussion and to assess how well 
prior concepts were understood by the class.

\subsection{\textit{Classical literature feedback}}
Each week students will be required to read a “classic” article on the current weeks’ 
topic and produce a short summary as well as two questions that arose during their 
reading of the article. Students are encouraged to bring up these questions during the
Tuesday class discussions.

\subsection{\textit{Recent literature article lead}}
Each student will be required to lead one Thursday discussion on recent literature. 
This will involve presenting the article 
and leading a discussion related to the article. Students must read some of the cited
literature integral to the study 
in order to answer relevant questions brought forth during the discussion.
The article will be chose by Dr. Smith, unless a different arrangement is made.
Discussion leads will be done in groups of 2-3 students.

\subsection{\textit{Recent literature article feedback}}
Students not leading the current week’s 
discussion will be required to produce a summary and 
develop two questions based on each week’s article.

\subsection{\textit{Literature Review}}
The primary semester project will be to produce a literature review on a topic 
of the student's choice.
Broadly, the review should address a question or problem related 
to plant ecophysiology and review the current state of knowledge on the topic.
The review should be forward thinking, in that it forms the
basis for understanding plant physiological processes moving forward.
The review should be novel in that it should not be similar to previously published
review papers.

Students will first develop a written proposal for their literature review and present 
their idea to the class. The class and instructors will provide feedback. Students will then produce and present 
their review to the class at the end of the semester. 

This project can be done
in groups of up to 3 students. Students are encouraged to receive help and guidance 
from the instructors as well as the class at large. 

The literature review will be assessed for completeness, breadth, originality, and presentation.
Students must have their project OKed by the instructor after the proposal and prior to
beginning the final project.

\section{Grading}
Participation and Engagement: 15\% \par
Mini-quizzes: 10\% \par
Classical literature feedback: 5\% \par
Recent literature lead: 15\% \par
Recent literature feedback: 5\% \par
Review idea proposal: 10\% \par
Review idea feedback: 5\% \par
Final review presentation: 10\% \par
Final review: 25\% \par

Grades will be made available on Blackboard. 
All grades posted at the end of the course will be final.
Please contact Dr. Smith if you feel your grade has been calculated incorrectly.

\section{Grading Scale}
A: $\geq$ 90\% \par
B: 80 – 90\% \par
C: 70 – 80\% \par
D: 60 – 70\% \par
F: $\leq$ 59.9\% \par

\section{Missing In-class Activities}
Students will be required to be in class to receive in-class activity points. 
Please contact Dr. Smith if you plan to miss class for a university function 
\textit{prior to class}. If class is missed due to an illness, 
please let Dr. Smith know as soon as possible (see COVID illness based absence policy below).

\subsection{Illness Based Absence Policy}
If at any time during this semester you feel ill, in the interest of your own health and 
safety as well as the health and safety of your instructors and classmates, you are 
encouraged not to attend face-to-face class meetings or events.  Please review the steps 
outlined below that you should follow to ensure your absence for illness will be excused. 
These steps also apply to not participating in synchronous online class meetings if you feel 
too ill to do so and missing specified assignment due dates in asynchronous online classes 
because of illness. If you are ill and think the symptoms might be COVID-19-related:
\begin{itemize}
	\item{Call Student Health Services at 806.743.2848 or your health care provider.  
	After hours and on weekends contact TTU COVID-19 Helpline at [TBA].}
	\item{Self-report as soon as possible using the Dean of Students COVID-19 webpage.
	This website has specific directions about how to upload documentation from a medical 
	provider and what will happen if your illness renders you unable to participate in 
	classes for more than one week.}
	\item{If your illness is determined to be COVID-19-related, all remaining 
	documentation and communication will be handled through the Office of the 
	Dean of Students, including notification of your instructors of the period of 
	time you may be absent from and may return to classes.}
	\item{If your illness is determined not to be COVID-19-related, please follow steps below.}
\end{itemize}

If you are ill and can attribute your symptoms to something other than COVID-19:
\begin{itemize}
	\item{If your illness renders you unable to attend face-to-face classes, participate 
	in synchronous online classes, or miss specified assignment due dates in asynchronous 
	online classes, you are encouraged to visit with either Student Health Services at 
	806.743.2848 or your health care provider.  Note that Student Health Services and 
	your own and other health care providers may arrange virtual visits.}
	\item{During the health provider visit, request a “return to school” note;}
	\item{E-mail the instructor a picture of that note;}
	\item{Return to class by the next class period after the date indicated on your note.}
\end{itemize}

\section{TTU COVID-19 Policy Reminders}
The Texas Tech University System has implemented a mandatory Facial Covering Policy to 
ensure a safe and healthy classroom experience. Current research on the COVID-19 virus 
suggests that there is a significant reduction in the potential for transmission of the 
virus from person to person by wearing a mask/facial covering that covers the nose and 
mouth areas. Because of the potential for transmission of the virus, and to be consistent 
with the University’s requirement, students in this class are to wear a mask/facial 
covering before, during, and after class. Observing safe distancing practices within 
the classroom by spacing out and wearing a mask/facial covering will greatly improve 
our odds of having a safe and healthy in-person class experience. Any student choosing 
not to wear a mask/facial covering during class will be directed to leave the class and 
will be responsible to make up any missed class content or work.

COVID-19-related links:
\begin{itemize}
	\item{Student Affair COVID-19 (\url{https://www.depts.ttu.edu/studentaffairs/SACOVID19.php})}
	\item{Student COVID-19 Protocol (\url{https://www.depts.ttu.edu/communications/emergency/coronavirus/provostdocs/Student_COVID-19_Flowchart_07-21-20.pdf})}
	\item{TTU Commitment (\url{https://www.ttu.edu/commitment/})}
\end{itemize}

\section{Special Considerations}
\subsection{Disabling Condition}
Any student who, because of a disability, may require special arrangements in order to 
Any student who, because of a disability, may require special arrangements in order to 
meet the course requirements should contact Dr. Smith as soon as possible to make 
any necessary arrangements. Students should present appropriate verification from Student 
Disability Services. Please note instructors are not 
allowed to provide classroom accommodations to a student until appropriate verification 
from Student Disability Services has been provided. For additional information, you may 
contact the Student Disability Services office at 335 West Hall or 806-742-2405.

\subsection{Religious Holy Days}
"Religious holy day" means a holy day observed by a religion whose places of worship are 
exempt from property taxation under Texas Tax Code §11.20.
A student who intends to observe a religious holy day should make that intention known 
in writing to the instructor prior to the absence. A student who is absent from classes 
for the observance of a religious holy day shall be allowed to take an examination or 
complete an assignment scheduled for that day within a reasonable time after the absence.
A student who is excused may not be penalized for the absence; however, the instructor 
may respond appropriately if the student fails to complete the assignment satisfactorily.

\section{Academic Integrity}
As stated in the Texas Tech University catalog, “The attempt of any students to present 
as their own work that they have not honestly performed is regarded by the faculty and 
administration as a serious offense and renders the offenses liable to serious 
consequences, possibly suspension.” This statement applies to cheating in whatever 
manner, including plagiarism.

\section{TTU Resources for Discrimination, Harassment, and Sexual Violence}
Texas Tech University is committed to providing and strengthening an educational, 
working, and living environment where students, faculty, staff, and visitors are 
free from gender and/or sex discrimination of any kind. Sexual assault, discrimination, 
harassment, and other Title IX violations are not tolerated by the University. 
Report any incidents to the Office for Student Rights and Resolution, 
(806)-742-SAFE (7233) or file a report online at 
\url{titleix.ttu.edu/students}. 

Faculty and staff members at TTU are committed to connecting you to resources on campus. 
Some of these available resources are: 
\begin{itemize}
	\item{TTU Student Counseling Center, 806-742-3674, \url{https://www.depts.ttu.edu/scc}. 
		Provides confidential support on campus.} 
	\item{TTU 24-hour Crisis Helpline, 806-742-5555. 
		Assists students who are experiencing a mental health or interpersonal violence 
		crisis. If you call the helpline, you will speak with a mental health counselor.} 
	\item{Voice of Hope Lubbock Rape Crisis Center, 806-763-7273, 
		\url{https://voiceofhopelubbock.org}.
		24-hour hotline that provides support for survivors of sexual violence.} 
	\item{The Risk, Intervention, Safety and Education (RISE) Office, 806-742-2110, 
		\url{https://www.depts.ttu.edu/rise/}. Provides a range of resources and support 
		options focused on prevention education and student wellness.} 
	\item{Texas Tech Police Department, 806-742-3931, 
		\url{http://www.depts.ttu.edu/ttpd/}. 
		To report criminal activity that occurs on or near Texas Tech campus.}
\end{itemize}

\section{LGBTQIA}
I identify as an ally to the lesbian, gay, bisexual, transgender, queer, intersex, 
and asexual (LGBTQIA) community, and I am available to listen and support you in an 
affirming manner. I can assist in connecting you with resources on campus to address 
problems you may face pertaining to sexual orientation and/or gender identity that could 
interfere with your success at Texas Tech. Please note that additional resources are 
available through the Office of LGBTQIA within the Center for Campus Life, 
Student Union Building Room 201, 
\url{www.lgbtqia.ttu.edu}, 806.742.5433.

\section{Online Classroom Civility}
Texas Tech University is a community of faculty, students, and staff sharing an 
expectation of cooperation, professionalism, respect and civility in all forms of 
university communication and business. This expectation applies to all interactions in a 
classroom setting where an exchange of ideas and creative thinking should be encouraged 
and where intellectual growth and development are fostered. As we consider ways in which 
we maintain a productive and cooperative online environment, many of the same standards 
from a face-to-face instruction transfer to the online setting. In this way, at the 
instructor’s discretion, disruptive behavior may result in disciplinary referrals pursuant 
to the Texas Tech University Code of Student Conduct. Students are expected to maintain 
online behaviors that are conducive to learning.

Examples of behavior that may be considered disruptive include:
\begin{itemize}
	\item{Disrupting the flow of a class session(s) by making off-topic comments.}
	\item{Enabling or participating in online classroom hijacking (“Zoombombing”) by 
	participating in online classroom streams without being enrolled in the course or 
	by sharing streaming classroom links with parties not enrolled in the course.}
	\item{Spamming, hacking, or using TTU or Blackboard platforms for commercial purposes.}
	\item{Cyberbullying or online harassment.}
	\item{Habitually interfering with or stopping instructional delivery}
\end{itemize}

\section{Creating Livable Futures}
This class is part of a campus-wide initiative called Creating Livable Futures, 
which is sponsored in part by the Texas Tech Center for Global Communication. 
As such, one of our objectives is to prepare you to communicate, 
in a fully interdisciplinary and global way, the challenges posed by pressing issues 
that speak to our collective wellbeing and sustainability. You will be asked to translate 
and communicate the work of leading thinkers on sustainability, and to expand discussing 
those materials through research experience and experiential learning.
These objectives will be met through discussion leads and the review paper. 

Your progress in communicating about global issues will be evaluated according to the 
Center for Global Communication rubric, so you will be invited to participate 
in one or more Creating Livable Futures activities outside of class that will 
complement class content. 
Planned Creating Livable Futures activities include participating in and attending 
speaker events and conferences, edit-a-thons, blogging and publication opportunities, 
student organizations, a book club, and even small scholarship opportunities for research. 

You’ll be informed of relevant opportunities and activities as they arise over 
the course of the semester.

\newpage

\section*{Schedule of Topics by Week}
Note: Lambers, Chapin III, and Pons (2008) pages in parentheses \par
18/01/21 – Introductions, semester planning, and goals \par
25/01/21 – Physiology’s role in ecology (pp. 1-8) \par
01/02/21 – Key physiological processes: 
photosynthesis, respiration, transpiration, translocation (pp. 11-203) \par
08/02/21 – Light (pp. 26-47, 237-238) \par
15/02/21 – Temperature (pp. 60-63, 127-129, 239-244) \par
22/02/21 – Water (53-57, 163-217) \par
01/03/21 – CO2 (pp. 87-90) \par
08/03/21 – Nutrients (pp. 58-59, 225-310) \par
15/03/21 – Growth and allocation (pp. 321-367) \par
22/03/21 – \textbf{Literature review proposal presentations} \par
29/03/21 – Life cycles, ontogeny, and phenology (pp. 375-398) \par
05/04/21 – Competition (pp. 505-527) \par
12/04/21 – Symbioses (pp. 522-524) \par
19/04/21 – Scaling from cells to canopies to ecosystems to the globe 
(pp. 247-253, 555-569) \par
26/04/21 – \textbf{Literature review presentations} \par
03/05/21 – \textbf{Literature review presentations} \par

\section*{General Weekly Schedule}
Generally, each Tuesday will consist of a lecture by Dr. Smith followed by a discussion
of a classical literature article. Students will turn in their classical literature
feedback at the end of Tuesday's lecture. Thursdays will generally begin with an in-class
mini-quiz and discussion. 
This will be followed by a discussion of a recent literature article and
(time permitting) an in-class activity.

} %end font selection

\end{document} 
